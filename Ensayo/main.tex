\documentclass[12pt]{article}
\usepackage[spanish]{babel}
\usepackage{amsmath}
\usepackage{graphicx}
\usepackage{hyperref}

\begin{document}

\begin{center}
\bf{\sc\Huge universidad de antioquia}
\end{center}
\vspace{120pt}
\begin{center}
\bf{\sc\Huge anderson alexis aristizabal garcia }\\
\end{center}
\vspace{200pt}
\begin{center}
\bf{\sc\Huge medellin}
\end{center}
\begin{center}
\bf{\sc\Huge 2020}\\
\end{center}\
\newpage



\begin{center}

\bf{\sc\Huge matematicas infalibles y el paso de la computacion moderna }\\
\end{center}
\large
Si enseñas un curso sobre teoría de números hoy en día, es probable que genere más interés entre las especialidades en informática que entre las matemáticas. Muchos se preocuparán poco por los enteros que se pueden expresar como la suma de dos cuadrados. En su lugar, preferirán aprender cómo Alice puede enviar un mensaje a Bob sin temor a que el espía Eve lo descifre. Sin duda, se sorprenderían de ver la teoría de los números descrita como una ''ciencia puramente teórica sin aplicaciones prácticas''o , aún más sin rodeos, como ''inútil''. Es con ironía que ahora leemos estos pronunciamientos, sabiendo que las semillas de su contradicción ya se habían sembrado. El trabajo que llevaría a la computadora digital moderna ya estaba en marcha. El gran avance teórico que condujo a la computadora moderna se remonta a 1936 cuando Alan Turing formuló un concepto muy original que eventualmente se llamaría la máquina de Turing. En ese momento, los proyectos para construir dispositivos informáticos más simples estaban a punto de comenzar.



\vspace{10PT}
A medida que la computadora evolucionó rápidamente de su función homónima, la lista de tareas que se le asignaron aumentó. Incluso las tareas que no implican un solo cálculo han sido asumidas por la computadora. Hoy en día, es probable que una revisión de un libro sea solicitada por comunicación informática, compuesta, investigada, revisada y escrita en una computadora, presentada por computadora y publicada para su acceso por una red mundial de computadoras. 


\vspace{15PT}
Incluso las ''pruebas de la impresora'' pueden llegar en forma de un archivo de computadora. La conversión de la teoría de números de una búsqueda ''inútil'' a una ciencia aplicada se debe en gran parte a una consecuencia especialmente irónica de la evolución de la computadora: para que podamos confiar de manera segura en la computadora para tareas no computacionales como comercio, comunicación, y archivar, primero debemos alistar la teoría de los números para frustrar el poder computacional de la computadora para descifrar. 

\vspace{15PT}
Al igual que el cambio radical en la teoría de los números que ocasionó, la metamorfosis de la computadora desde el generador de números hasta la máquina lógica de uso múltiple ha sido una transformación profunda que ahora se da por sentado, pero originalmente no era transparente. Aiken, por ejemplo, no reconoció la transición en progreso. ''Si resultará'', escribió en 1956, ''que las lógicas básicas de una máquina diseñada para la solución numérica de ecuaciones diferenciales coinciden con las lógicas de una máquina destinada a hacer facturas para una tienda por departamentos, consideraría esto como la coincidencia más asombrosa que jamás haya encontrado''.


\vspace{15PT}
Que Turing había enclavado el futuro de la informática antes de que todos los demás puedan verse en varias de sus declaraciones, de las cuales lo siguiente es típico: ‘’No habrá ninguna alteración interna que hacer, incluso si deseamos cambiar repentinamente del cálculo los niveles de energía del átomo de neón para la enumeración de grupos de orden 720 ‘’. En lo expresó de esta manera: ‘’No necesitamos tener una infinidad de máquinas diferentes que realicen diferentes trabajos. Una sola será suficiente’’.

\vspace{15PT}
 Turing no se refirió a esta máquina individual por el nombre inapropiado que otros con visiones más estrechas ya estaban usando: la llamó la máquina universal, fue la concepción de Turing de la máquina universal lo que influyó. 

\vspace{15PT}
“Un maestro no podra enseñar nunca en forma verdadera, si él mismo no está en actitud de aprender. Una lámpara no puede encender otra lampara si no tiene encendida su propia llama”. Rabindranath Tagore

\vspace{15PT}
Georg Cantor, el fundador de la teoría de conjuntos, demostró en la década de 1870 que no todos los conjuntos infinitos son iguales: en particular, el conjunto de números enteros es 'más pequeño' que el conjunto de todos los números reales, también conocido como el continuo. (Los números reales incluyen los números irracionales, así como los racionales y los enteros). Cantor también sugirió que no puede haber conjuntos de tamaño intermedio, es decir, más grandes que los enteros, pero más pequeños que el continuo. Pero no pudo probar esta hipótesis del continuo, y tampoco muchos matemáticos y lógicos que lo siguieron.

\vspace{15PT}
Sus esfuerzos fueron en vano. Un resultado de 1940 de Gödel (que fue completado en la década de 1960 por el matemático estadounidense Paul Cohen) mostró que la hipótesis del continuo no puede probarse como verdadera o falsa a partir de los axiomas estándar, las afirmaciones consideradas verdaderas, de la teoría de conjuntos, que comúnmente se toman como la base de todas las matemáticas.

\vspace{15PT}
El trabajo de Gödel y Cohen sobre la hipótesis del continuo implica que pueden existir universos matemáticos paralelos que sean compatibles con las matemáticas estándar: uno en el que la hipótesis del continuo se agrega a los axiomas estándar y, por lo tanto, se declara verdadera, y otro en el que se declara falso.

\newpage
\bf{\sc\Huge bibliografia }\\
\begin{itemize}
\item  \url{https://www.bbvaopenmind.com/ciencia/matematicas/asi-termino-el-sueno-de-las-matematicas-infalibles/?utm_source=materia&utm_medium=facebook&tipo=elabora&cid=soc:afl:fb:----materia:--:::::::sitlnk:materia:&fbclid=IwAR2oBVdboWbxmrIAqVbiA2hZrBTjcvLSl6RQ5Fh5UQX2ZvvybK8MlpGi3gM}
    \item \url{http://mathshistory.st-andrews.ac.uk/Biographies/Cantor.html}
    \item \url{https://www.scientificamerican.com/article/what-is-russells-paradox/}
    \item \url{https://plus.maths.org/content/alan-turing-ahead-his-time}
    \item \url{https://www.bbvaopenmind.com/ciencia/grandes-personajes/la-extraordinaria-historia-de-alan-turing/}

\end{itemize}
\end{document}
